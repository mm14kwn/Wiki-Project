\documentclass{report}
  \setcounter{tocdepth}{0}
  \setcounter{secnumdepth}{4} 
  \usepackage{lipsum}
  \usepackage[round]{natbib}
\usepackage{hyperref}
  \usepackage{titletoc}
  \begin{document}
  
  \title{Air-Sea Interactions\\SOEE5680M Wiki Project} 
  \author{Kieran Newman\\200901399}
  \date{\small\today}
  \maketitle
  
  %tell tex4ht to make main toc show only chapters 
  %thanks to Radhakrishnan CV for this solution
  \ifdefined\HCode
  \Configure{tableofcontents*}{chapter} 
  \fi
  
  \tableofcontents
  
  \ifdefined\HCode
  \TocAt{chapter,section} %show section only in chapters TOC 
  \TocAt{section,subsection} %show subsection only in sections TOC
  \fi
  
  %---------------------
  \chapter{Introduction and Definitions} 

  \ifdefined\HCode
  \else
  {
  \startcontents[chapter]
  \printcontents[chapter]{}{1}{\setcounter{tocdepth}{1}} 
  }
  \fi
    
  
  %---------------------
   
  \ifdefined\HCode
  \PauseCutAt{section}
  \fi
  
  \ifdefined\HCode
  \else
  {
  \stopcontents[chapter]
  } 
  \fi
 \subsection{Introduction} 
\label{intro}
 Air-Sea interactions on various scales have important impacts on various atmospheric and oceanic processes, such as \hyperref[hurricanes]{hurricanes}, \hyperref[circulation]{ocean circulation}, large scale atmospheric motions and monsoon cycles.
There are three main quantities transferred between atmosphere and ocean, namely  \hyperref[momentumtransfer]{momentum}, \hyperref[heattransfer]{heat} and \hyperref[masstransfer]{mass}.
The laws governing these transfers are generally implicit relationships between the flux of a quantity, the related forces in the flow, and various properties of the system.
This implicit relationship is represented in \citet{csanady04} by the equation
\begin{equation}
f\left(\mbox{Flux},\mbox{Force},\sum^n_{i=1} X_i \right) =0,
\label{eq:fluxforce}
\end{equation}
for $n$ variables $X_i$ that influence the transfer.
This \hyperref[eq:fluxforce]{relationship} can be reworked with knowledge of the \hyperref[def:buckingham]{Buckingham $\pi$ Theorem} to give
\begin{equation}
f\left(\sum^m_{i=1} N_i \right) =0,
\label{eq:fluxforce2}
\end{equation}
for $m$ nondimensional combinations of the force, flux and variables.
If the flux or force appears in one $N_i$ only, then this relationship can be used by setting this as the response variable, with all others set as explanatory variables.
Variations on \hyperref[eq:fluxforce2]{this equation} will be used in later sections to describe the mechanics of various local and \hyperref[def:bulk]{bulk} transfers.
\section{Definitions}
\label{definitions}
\begin{description}
\item[Buckingham $\pi$ Theorem] First proved by \citeauthor{bertrand78} in \citeyear{bertrand78}, and expanded on by \citet{rayleigh92} and \citet{buckingham14}, this theorem states that if some physical process is described by $n$ variables, of which $k$ are repeating (generally linear dimensions, density and velocity in fluid dynamics), then there exists $n-k$ non-dimensional, independent parameters that can be used to describe the process, in some combination.\label{def:buckingham}
\item[Laminar Flow] If a fluid flows purely in parallel layers, with no mixing between, then it is known as laminar flow \citep{batchelor00}.\label{def:laminar}
\item[Turbulent Flow] The opposite of \hyperref[def:laminar]{laminar flow}, turbulent flow typically is chaotic, with rapid pressure and velocity changes in time and space \citep{batchelor00}.\label{def:turbulent}
\item[Bulk Transfer] Generally in air-sea interactions, there is transfer of momentum/heat/etc both from air to sea and from sea to air. The inclusion of \hyperref[def:boundarylayer]{boundary layers} further complicates the use of local descriptions of this transfer. To this end, we generally formulate laws in terms of the difference in a quantity across the boundary layer, and the flux across the air/sea interface \citep{csanady04}. \label{def:bulk}
\end{description}

\chapter{Momentum Transfer}
\label{momentumtransfer}
    \lipsum[10]
  
    \section{Surface Layers}
\label{surfacelayers}
    \lipsum[1-2]

   \subsection{Turbulence}
\label{momturbulence} 
    \lipsum[1-2]

    \section{Charnock's Relation}
\label{charnock}
    \lipsum[1-2]
  
    \section{Surface Roughness}
\label{surfaceroughness}
    \lipsum[1-2]

    \section{Dissipation}
\label{dissipation}
    \lipsum[1-2]  	

    \section{Wind Waves}
\label{windwaves}
    \lipsum[1-2]

    \section{Wave Breaking}
\label{momwavebreaking}
    \lipsum[1-2]

\chapter{Heat Transfer}
\label{heattransfer}
    \lipsum[10]
  
    \section{Sensible and Latent Heat Fluxes}
\label{heatflux}
    \lipsum[1-2]

\subsection{Wind Induced Surface Heat Exchange (WISHE)}
\label{wishe}
   
    \section{Diffusion}
\label{diffusion}
    \lipsum[1-2]

    \section{Hot Towers}
\label{hottowers}
    \lipsum[1-2]

\subsection{Hurricanes}
\label{hurricanes}

    \section{Convection}
\label{convection}
    \lipsum[1-2]

    \section{Ocean Circulation}
\label{circulation}
    \lipsum[1-2]

    \section{CAPE}
\label{cape}
    \lipsum[1-2]

\chapter{Mass Transfer}
\label{masstransfer}
    \lipsum[10]
  
    \section{Gas Transfer}
\label{gastransfer}
    \lipsum[1-2]
   
    \section{Aerosols}
\label{aerosols}
    \lipsum[1-2]

    \section{Turbulence}
\label{massturbulence}
    \lipsum[1-2]
  
    \section{Wave Breaking}
\label{masswavebreaking}
    \lipsum[1-2]

\chapter{References}
\bibliographystyle{dcu}
\bibliography{kwnrefs}
  \end{document}
